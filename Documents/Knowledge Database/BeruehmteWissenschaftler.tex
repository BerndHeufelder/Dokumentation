\part{Berühmte Wissenschaftler}
\addsec{15'th century}
\subsection{Leonardo da Vinci 			1452-1519}
\subsection{Nikolaus Kopernikus 		1473-1543}
\addsec{16'th century}
\subsection{Galileo Galilei 			1564-1642}
\subsection{Johannes Keppler 			1571-1630}
\addsec{17'th century}
\subsection{Isaac Newton 				1643-1727}
\addsec{18'th century}
\subsection{Carl Friedrich Gauß 		1777-1855}
\subsection{Michael Faraday 			1791-1867}
\addsec{19'th century}
\subsection{Charles Darwin 			1809-1882}
\subsection{Gustav Robert Kirchhoff    1824-1887}
\subsection{James Clerk Maxwell 		1831-1879}
\subsection{Thomas Alva Edison 		1847-1931}
	Wer der Erfinder der Glühbirne ist, ist nach wie vor ein Rätsel. Anscheinend soll der deutsche Uhrenmacher Heinrich Göbel im Jahre 1854 die erste Glühlampe gebaut und erfunden haben. Jedoch verfügte dieser weder über das Know-How noch die technischen Geräte um ein derartiges Projekt fertigzustellen. 1879 hat Edison eine Kohlefaden-Lampe gebaut und darauf ein Patent erhalten. Wer aber die Idee dazu ist unbekannt.
	\begin{itemize}
		\item Thomas Edison meldete für über tausend Erfindungen Patente an
		\item Edison galt als skrupelloser Geschäftsmann
		\item Er versuchte den Gleichstrom bei der Elektrifizierung für seinen eigenen Profit durchzusetzen, gewonnen hat der Wechselstrom von Westinghouse. Er richtete sogar öffentliche Tiere durch Wechselstrom hin um zu zeigen, dass dieser gefährlich sei und sein Gleichstrom nicht. Als Höhepunkt brachte er einen Elefanten um.
	\end{itemize}
\subsection{Nikola Tesla 				1856-1943}
	Ausbildung: Technische Universität Graz\\
	Herkunft: Serbien\\
	In Österreich bis 1891 gelebt, danach bis zum Tod in Amerika. \\\\
	\textbf{Erfindungen:}
	\begin{itemize}
		\item Drehstrom-Asynchronmaschine
		\item rotierendes magnetisches Feld
		\item Tesla-Spule (el. Energie durch Luft übertragen: hochspannung, niedrgier Strom, hohe Frequenz)
		\item erstes ferngesteuertes Boot
	\end{itemize}
	
\subsection{Heinrich Hertz 			1857-1894}
\subsection{Max Planck 				1858-1947}
\subsection{Albert Einstein 			1879-1955}
\subsection{Niels Bohr 				1885-1962}
\addsec{20'th century}
\subsection{Werner Heisenberg 			1901-1976}
\subsection{Stephen Hawking 			1942-heute}