\part{General}

	\textbf{Overview of to parameters to take into account}\label{param}
		\begin{itemize}
			\item Moving mass
			\item Maximum rotational speed
			\begin{itemize}
				\item of motor 
				\item regarding electrical frequency of output stage
				\item of spindle or belt if present
			\end{itemize}
			\item Maximum acceleration
			\item Mechanical losses
			\begin{itemize}
				\item Static friction
				\item Viscous friction
				\item Efficiency (Spindel, Belt, ...)
			\end{itemize}
			\item Required Accuracy
		\end{itemize}
	\leavevmode\\
	
	\textbf{General method}
		\begin{enumerate}
			\item Estimate system parameters from section \ref{param}
			\item Calculate required torque for desired trajectory 
			\item Compare required torque over speed against the manufacturers reference curve
		\end{enumerate}
	\leavevmode\\
	
	\textbf{List of occured issues}\leavevmode\\\\
		\begin{tabular}{p{4cm}p{12cm}}
			\rowcolor{gray!10!white}
			\textbf{Efficiency of indirect linear spindel axis} & 
			For fully assembled indirect-linear-spindle-drives from Nanotec a force over speed curve is provided. The spindle efficiency is already taken into account in the force-speed curve.
			\\\hline&\\
			\rowcolor{gray!00!white}
			\textbf{Max rotational Speed of Stepper Motors} & 
			Stepper motors are usually not built for high speed applications. Therefore it is understandable that the torque-reference-curve is only correct for rotational speeds up to 700rpm. Above that speed, the from Nanotec given torques are usullay not achieved. \\\hline&\\
		\end{tabular}