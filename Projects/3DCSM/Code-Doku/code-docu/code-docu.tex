\documentclass[10pt,a4paper]{article}
\usepackage[utf8]{inputenc}
\usepackage{ngerman}
\usepackage{graphicx}
\usepackage{amsmath}
\usepackage{amssymb}
\usepackage{amsfonts}
\usepackage{relsize}
\usepackage{media9} 
\usepackage{tikz}
\usepackage[overload]{empheq}
\usetikzlibrary{arrows, matrix, trees, mindmap, positioning}
\usepackage{arydshln}
\usepackage{animate}
\usepackage{anyfontsize}
\usepackage{bm}
\usepackage{wrapfig}
\usepackage{listings}
\usepackage{changepage}
\newcommand{\tikzfnt}{13}
\newcommand{\tfcn}[1]{\textit{\textbf{#1}}}
\newcommand{\cbox}[1]{\colorbox{green!50!black!30}{#1}}

% Titlepage settings
\title{3DCSM - Body Handling\\\normalsize Simulations-Code Dokumentation\\\small Code-Struktur und Simulationsablauf}
\author{Bernd Heufelder}
\date{\today}

\begin{document}
	\maketitle
	\newpage
	\tableofcontents
	\newpage
	\section{Bewegungsgleichung erstellen}
		Die komplette Erzeugung der Bewegungsgleichungen erfolgt im File \textit{eq\_of\_motion.py}. Die Bewegungsgleichungen werden als Binärfiles im Format \textit{.pickle} abgespeichert. Die benötigten Parameter werden aus dem File \textit{param.py} importiert.
	\section{Simulation des Modells}
		Gestartet wird die Simulation über das File \text{simulation.py} durch die Klasse \textsc{Simulation}.
		\paragraph{\cbox{oSim = Simulation():}}
		 In der \textit{\_\_init\_\_()} wird zuerst das Modell \textit{model\_full()} instanziert, dann ein Trajektorienobjekt entweder anhand von externen Textfiles oder durch Vorgabe von Start- und Endwerten erzeugt. Die Zustandsvariablen zum Zeitpunkt Null werden anhand des Trajektorienobjektes gesetzt. 
		 \begin{enumerate}
		 	\item \tfcn{model\_full.\_\_init\_\_():} Dieses wiederum instanziert das mechanische und elektrische Modell. Dafür werden die Files \textit{model\_full.py, model\_mech.py, model\_el.py} verwendet. 
		 	\begin{adjustwidth}{1cm}{}
		 		\tfcn{model\_mech.\_\_init\_\_():} Hier werden die Binärfiles der Bewegungsgleichungen geladen und in dieser Klasse als Properties gespeichert. Zusätzlich wird als Property des mechanischen Modells der PID-Regler der Trajektorie aus dem File \textit{control.py} instanziert.
		 		\begin{adjustwidth}{1cm}{}
		 			\tfcn{PID\_traj.\_\_init\_\_():} Hier werden die Regelparameter für den Trajektorienregler als Properties der \textsc{PID\_traj} Klasse definiert.
		 		\end{adjustwidth}
		 	\end{adjustwidth}
	 		\begin{adjustwidth}{1cm}{}
	 			\tfcn{model\_el.\_\_init\_\_():} Hier wird nur der Stromregler der Motoren über die Klasse \textsc{PID\_motor} instanziert.
	 			\begin{adjustwidth}{1cm}{}
	 				\tfcn{PID\_motor.\_\_init\_\_():} Hier werden die Regelparameter für den Trajektorienregler als Properties der \textsc{PID\_motor} Klasse definiert.
	 			\end{adjustwidth}
	 		\end{adjustwidth}
 			\item \tfcn{Trajektorienobjekt erstellen}\\
 			\item\tfcn{Initialwerte des Modells aus der Trajektorie setzen}\\
 			\item\tfcn{Arrays zur Datenspeicherung erstellen}
		 \end{enumerate}
		 
		\paragraph{\cbox{oSim.integrate():}}
			Hier werden die Bewegungsgleichungen anhand eines ODE-Solvers von den Initialwerten aus zum Zeitpunkt Null bis zum Endzeitpunkt iterativ gelöst. Dem Solver wird als Systemfunktion die Funktion \textit{system\_equations(t, state, oTraj)} aus dem File \textit{model\_full.py} zugewiesen. Dem Solver kann eine Schrittweite $ dt $ angegeben werden, welche die zeitliche Differenz zwischen zwei Integrationsschritten vorgibt. Der Solver hat aber ebenso eine Variable $ nsteps $, welche die maximale Anzahl an Zwischenintegrationsschritten vorgibt. Das heißt, der Solver integriert innerhalb eines $ dt-Intervalls $ maximal nochmals $ nsteps $-mal um eine gewünschte Toleranz des Ergebnisses zu gewährleisten. 
		\paragraph{\cbox{oSim.save\_results():}}
		\paragraph{\cbox{oSim.analyse\_mech\_eigenfreq():}}
	
	\section{Ergebnisse darstellen und visualisieren}

\end{document}