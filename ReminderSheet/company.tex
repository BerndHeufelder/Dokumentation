\section{Firmenspezifisch}
	\subsection{Firmenstruktur}
		\subsubsection{Software}
			\paragraph{BIB}
				Softwareabteilung, besthend aus OMA und Betriebssystem für die Maschine (Linux).
				\subparagraph{OMA}\leavevmode\\
					Hardwareabstraktion auf Sofwareebene; Applikation erhält möglichst einheitliche Objekte für unterschiedliche Hardware. Die Applikation verwendet diese Hardware dann entsprechend und programmiert damit einen Prozessablauf. Kommunikation der HW über CAN-Bus und PC. OMA ist nicht echtzeitfähig (z.b. Rückmeldung nach Bewegungsende). Bildverarbeitung ist eigenständig und nicht Teil der OMA. Auch wird durch die OMA die Topologie der Hardware, also welche HW an welchem Knoten hängt, verschleiert. \\\\
					Link zur Doku: \hyperlink{dsf}{https://confluence.besi.com/display/MA/interferometric+systems+RENISHAW}
			\paragraph{Applikation}
				Programiert einen Prozessablauf und seine Bewegungen. Ebenso wird die Benutzeroberfläche auf der Maschine von der Applikation bereitgestellt.
	\subsection{Ansprechpersonen}
		\begin{tabular}{l|p{14cm}}
			\rowcolor{gray!10!white}
			\textbf{Simon Holzknecht} & Chef Software Tester, Ansprechpartner zum Reservieren von Maschinen \\
			\rowcolor{gray!00!white}
			\textbf{Michael Kirchler} & Zuständiger System Engineer für 2k2 \\
			\rowcolor{gray!10!white}
			\textbf{Bernd Jandl} & Manager Machine Integration Software\\
			
		\end{tabular}